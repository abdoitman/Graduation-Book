\section*{PDSCH overview \& Motivation}

The Physical Downlink Shared Channel (PDSCH) is a crucial component of 5G wireless communication systems that enables high-speed data transmission with low latency. As a team, we were intrigued by the potential of 5G technology to revolutionize the way we communicate and connect with each other. We were particularly interested in the PDSCH channel, as it plays a critical role in achieving the high data rates and low latency that are essential for applications such as virtual reality, autonomous driving, and the Internet of Things (IoT). \\
We were also motivated by the technical challenges involved in simulating the PDSCH channel accurately. 5G is a complex and evolving technology that requires a deep understanding of wireless communication systems, signal processing, and coding theory. Our project aimed to develop a MATLAB based simulation model that could capture the unique characteristics of PDSCH and provide insights into its performance under different conditions. \\

In this Part, we will delve into the technical details of PDSCH and explain how our simulation model was designed and implemented. We will also discuss the results of our simulation experiments and highlight the insights we gained from them. Our hope is that this work will contribute to the ongoing efforts to improve the performance and reliability of 5G networks. \\
Designing and implementing a simulation model for PDSCH requires a detailed understanding of the 5G standard specifications, as well as knowledge of wireless communication systems, signal processing, and coding theory. Here is an overview of the steps we took to design and implement our simulation model: \\
Our simulation model for PDSCH in 5G wireless communication systems consists of several key blocks that are critical to achieving high-speed data transmission with low latency. These blocks include channel coding, MIMO setups, reference signals, and channel estimation. \\

The channel coding block is responsible for protecting the data from errors that may be introduced during wireless transmission. We simulated both linear block codes and LDPC coding to evaluate their performance under different conditions. \\
The MIMO setups block allows for multiple antennas to be used at both the transmitter and receiver, which can improve the quality and reliability of wireless communication. We simulated several MIMO setups, including SISO, SIMO, and MIMO, to gain insights into the performance of PDSCH under different conditions.\\
The reference signals block provides important information about the wireless channel, such as channel state information and data demodulation and decoding. We focused on two types of reference signals: CSI-RS and DMRS, which are used for channel estimation, synchronization, and data demodulation. \\

Finally, the channel estimation block is responsible for accurately estimating the wireless channel conditions and compensating for any distortion or interference that may be present. We spent a significant amount of time on channel estimation, which is essential for accurately simulating the performance of PDSCH under realistic conditions. \\
In the following sections, we will discuss each of these blocks in detail, including the technical details of how they were implemented and the results of our simulation experiments.
